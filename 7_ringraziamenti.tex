\documentclass[./main.tex]{subfiles}

\begin{document}

\chapter*{Ringraziamenti}
In questo capitolo vorrei ringraziare tutte le persone che in qualche modo hanno influenzato 
e sostenuto il mio percorso di studi universitario e la stesura di questa tesi.
Non sono mai stato particolarmente sentimentale e sicuramente non sono bravo a scrivere di questi argomenti,
quindi ho deciso di delegare questa parte alla musica. 
Ad ognuno di voi dedico un brano, tra quelli che ascolto quotidianamente, che in qualche modo mi ricorda voi.

\begin{itemize}
    \item Un ringraziamento speciale va ai miei relatori, il Prof. \textbf{Massimo Benerecetti} e il Prof. \textbf{Fabio Mogavero}.
    Sono stati loro a darmi l'opportunità di svolgere questo lavoro e mi hanno sempre supportato e guidato.
    I nostri incontri sono sempre stati molto piacevoli e stimolanti.
    Ringrazio anche il mio collega di tirocinio, \textbf{Francesco}, 
    senza il quale probabilmente sarei impazzito durante il reverse engineering di Vampire :).
    Ai miei relatori dedico il brano \textit{BWV 1052} di \textit{Johann Sebastian Bach}. 
    Un brano tecnicamente complesso e strutturato, ma allo stesso tempo elegante, armonioso e comprensibile.
    All'intero gruppo del tirocinio dedico, invece, il brano \textit{BWV 1065} sempre di \textit{Johann Sebastian Bach}.
    Il concerto per quattro clavicembali mi ricorda in qualche modo i nostri incontri a quattro voci.

    \includegraphics[width=3.5cm]{images/7_ringraziamenti/professori.png}
    \includegraphics[width=3.5cm]{images/7_ringraziamenti/professori2.png}

    \item Un ringraziamento va alle mie due gatte, \textbf{Zelda} e \textbf{Mipha}, che mi hanno tenuto compagnia in questi anni di studio.
    In particolare Zelda ha studiato con me ogni esame, 
    sedendosi sui miei appunti sperando forse di apprendere qualcosa per osmosi.
    Anche solo per questo meriterebbe una laurea ad honorem.
    Durante il periodo del Covid ha anche partecipato attivamente a buona parte delle lezioni online e agli esami.
    Sostenere un esame con un gatto in braccio è un'esperienza che consiglio a tutti.
    Mipha, invece, è arrivata troppo tardi per seguire tutte le lezioni, 
    ma mi ha tenuto compagnia durante tutto il periodo di tirocinio e di stesura della tesi.
    A loro dedico il brano \textit{Humoresque} di \textit{Antonín Dvořák}.
    Un brano allegro e giocoso, come loro.

    \includegraphics[width=3.5cm]{images/7_ringraziamenti/gatti.png}

    
    \item Un ringraziamento speciale va al mio amico storico e collega di studi, \textbf{Claudio}, che mi ha accompagnato in questo percorso di studi.
    Senza il suo incoraggiamento probabilmente starei ancora procrastinando per preparare e sostenere l'esame di Reti,
    così come buona parte degli esami noiosi.
    Un ringraziamento va anche a \textbf{Mary}, che con la sua compagnia ha reso più piacevoli le mattinate in treno per andare a lezione.
    A loro dedico il brano \textit{BWV 1060} di \textit{Johann Sebastian Bach}.
    
    \includegraphics[width=3.5cm]{images/7_ringraziamenti/claudio.png}

    \item Un ringraziamento va ai miei amici \textbf{Umby45} e \textbf{Emanuele}, che nell'ultima sessione di esami mi hanno fatto compagnia durante
    il corso di Geometria, un corso che sembrava una noia estenuante ma grazie a loro è diventato un'esperienza divertente.
    A loro dedico il brano \textit{Divertimento K. 334} di \textit{Wolfgang Amadeus Mozart}.

    \includegraphics[width=3.5cm]{images/7_ringraziamenti/umby.png}

    \item Un ringraziamento va al mio amico e collega di studi, \textbf{Stusio}, che è strato un grande compagno
    di avventure e disavventure durante questi anni di università.
    % Ricordo ancora di quando facemmo l'esame di Fisica insieme fino a tarda notte.
    A lui dedico il brano \textit{Odissea Veneziana} di \textit{Rondò Veneziano}.

    \includegraphics[width=3.5cm]{images/7_ringraziamenti/stusio.png}

    \item Un ringrazio va a \textbf{Marzia} e \textbf{Giulio}, che confermano il fatto che anche le piccole cose contano.
    Una delle parti più pesanti dell'università è il viaggio in treno per andare a lezione.
    Marzia senza neanche saperlo ha reso più piacevoli queste mattinate facendomi scoprire Settimana Sudoku,
    un libricino che porto sempre con me e che mi ha tenuto compagnia in questi viaggi.
    Ringrazio anche Giulio per avermi regalato una settimana di distrazione prima dell'inizio della stesura della tesi.
    Può sembrare stupido, ma il fatto che \textit{Kaalub} mi abbia detto di non preoccuparmi per quanto riguarda i tempi di consegna, 
    mi ha veramente rasserenato e mi ha permesso di affrontare la scrittura con più tranquillità.
    A loro dedico il brano \textit{Bourrée} di \textit{Jethro Tull}.

    \includegraphics[width=3.5cm]{images/7_ringraziamenti/marzia_giulio.png}

    \item Un ringraziamento va a mia \textbf{Nonna}, che mi ha sempre sostenuto e incoraggiato. 
    Nessuno più di lei (me compreso) aveva così tanta ansia per ogni esame che ho sostenuto.
    Spero che adesso sia un po' più tranquilla.
    A lei dedico il concerto per Flauto di \textit{Wolfgang Amadeus Mozart}.

    \includegraphics[width=3.5cm]{images/7_ringraziamenti/nonna.png}
    
    \item Un ringraziamento speciale va a mia \textbf{Madre}, che è sempre stata la mia più grande sostenitrice,
    paziente e orgogliosa di me in ogni occasione.
    Spesso sento dire dai miei amici che vorrebbero una madre come la mia, ma io non ho mai desiderato una madre diversa.
    A lei dedico il mio brano preferito, \textit{Il Pastor Fido} di \textit{Georg Friedrich Händel}.
    
    \includegraphics[width=3.5cm]{images/7_ringraziamenti/mamma.png}

    \item Un ringraziamento speciale va a \textbf{Stefano}, che insieme a mia Madre è stato il mio più grande sostenitore.
    Ha influenzato positivamente e in modo significativo la mia vita e il mio percorso di studi. 
    È grazie a lui che ho scoperto e portato avanti varie passioni, tra cui l'informatica.
    A lui dedico il brano \textit{BWV 1064} di \textit{Johann Sebastian Bach}.

    \includegraphics[width=3.5cm]{images/7_ringraziamenti/stefano.png}

    \item Un ringraziamento inaspettato va a mio \textbf{Padre}, che nonostante il nostro pessimo rapporto, 
    ha avuto grande influenza sui miei interessi.
    Anni fa mi consigliò di leggere il libro \textit{Gödel, Escher, Bach} di \textit{Douglas Hofstadter}.
    Durante la lettura sentii che per capirlo a pieno avrei dovuto saperne di più sugli argomenti trattati
     e per questo iniziai a seguire il corso di Logica.
    La spinta a scrivere una tesi su tali argomenti deriva proprio da questi eventi.
    A lui dedico il Brano delle danze slave \textit{Op. 72 n. 2} di \textit{Antonín Dvořák}.
    
    \includegraphics[width=3.5cm]{images/7_ringraziamenti/padre.png}

    \item Un ringraziamento speciale va a \textbf{Conny}, il mio più grande sostegno emotivo.
    Le parole non bastano per esprimere la gratitudine che provo nei tuoi confronti.
    A lei dedico il brano \textit{Romance Op. 11} di \textit{Antonín Dvořák}.
    
    \includegraphics[width=3.5cm]{images/7_ringraziamenti/conny.png}


\end{itemize}




\end{document}