\documentclass[./main.tex]{subfiles}


\begin{document}
\chapter{Prerequisiti}

In questo capitolo verranno descritte le nozioni di base necessarie 
per comprendere il lavoro svolto. 
In particolare, verranno introdotti i concetti di logica proposizionale e 
del primo ordine, definita come estensione della prima. Nell'ultimo paragrafo del capitolo 
verrà descritto in che modo le formule di logica del 
primo ordine possono essere rappresentate in un formato di file, per poi essere 
processate come input da un theorem prover. 
Lo scopo di questo capitolo è quello di accennare la teoria logica utilizzata nell'implementazione di vampire 
e della procedura di decisione per i Binding-Fragments. Perciò, verranno date per scontate nozioni di teoria degli insiemi,
algebra booleana e teoria dei linguaggi formali.



\section{Logica Proposizionale}

% ---------------------- FORMULE ----------------------
\subsection{Formule}
Sia $\Sigma_c = \{c_1, c_2, ...\}$ un insieme di simboli di costante, 
$\Sigma = \{ \land, \lor, \lnot, (, ), \top, \bot\} \cup \Sigma_c$ è detto alfabeto della logica proposizionale. 
%Dato un insieme di simboli $A$, l'insieme $A^*$ è definito come l'insieme delle stringhe finite su $A$.
Con queste premesse possiamo definire come formule della logica proposizionale il linguaggio generato dalla
grammatica Context Free seguente:
$$
\varphi  := \top \mid \bot \mid C \mid \lnot \varphi \mid (\varphi \land \varphi) \mid (\varphi \lor \varphi)
$$

Dove $C \in \Sigma_c$ è un simbolo di costante. Con la funzione $const(\gamma) \rightarrow \Sigma_c$ si indica la funzione che
associa a ogni formula $\gamma$ l'insieme dei suoi simboli di costante.
Vengono inoltre introdotti i seguenti simboli come abbreviazioni:

\begin{itemize}
    \item $(\gamma \Rightarrow \kappa)$ per $(\lnot \gamma \lor \kappa)$
    \item $(\gamma \Leftrightarrow \kappa)$ per $((\gamma \Rightarrow \kappa) \land (\kappa \Rightarrow \gamma))$
    \item $(\gamma \oplus \kappa)$ per $\lnot(\gamma \Leftrightarrow \kappa)$
\end{itemize}

È possibile rappresentare una qualunque formula attraverso il proprio albero di derivazione. Questo albero 
verrà chiamato in seguito anche \textit{albero sintattico} della formula. Ad esempio, la formula 
$(c_1 \land c_2) \lor \lnot c_3$ può essere rappresentata dal seguente albero sintattico:

\begin{center}
    \begin{tikzpicture}[level distance=1cm,
        level 1/.style={sibling distance=3cm},
        level 2/.style={sibling distance=1.5cm}]
        \node {$\lor$}
          child { node {$\land$}
            child {node {$c_1$}}
            child {node {$c_2$}}
          }
          child { node {$\lnot$}
            child {node {$c_3$}}
          };
    \end{tikzpicture}
\end{center}


La radice dell'albero è detta \textit{connettivo principale}. Per compattezza, grazie alla proprietà associativa di $\land$ e $\lor$, è possibile omettere le parentesi, es. 
$(c_1 \land (c_2 \land (c_3 \land c_4))) \lor c_5$ può essere scritto come $(c_1 \land c_2 \land c_3 \land c_4) \lor c_5$. 
Allo stesso modo, nell'albero sintattico della formula è possibile compattare le catene di $\land$ e $\lor$ come figli di un unico nodo:

\begin{center}
    \begin{tikzpicture}[level distance=1cm,
        level 1/.style={sibling distance=2.5cm},
        level 2/.style={sibling distance=1cm}]
        \node {$\lor$}
          child { node {$\land$}
            child {node {$c_1$}}
            child {node {$c_2$}}
            child {node {$c_3$}}
            child {node {$c_4$}}
          }
            child { node {$c_5$}};
    \end{tikzpicture}
\end{center}

Questa è una caratteristica molto importante, in quanto non solo permette di risparmiare inchiostro, ma consente di vedere
$\land$ e $\lor$ non più come operatori binari ma come operatori n-ari. A livello implementativo, ciò si traduce in un minor
impatto in memoria, visite all'albero più veloci e algoritmi di manipolazione più semplici. Si consideri ad esempio di voler ricercare la 
foglia più a sinistra nell'albero di derivazione della seguente formula $(( ... (((c_1 \land c_2) \land c_3) \land c_4) \land ... )\land c_n)$.
Senza compattazione, l'algoritmo di ricerca impiegherebbe $O(n)$ operazioni, mentre con la compattazione $O(1)$.

% ---------------------- END FORMULE ----------------------



       
    \subsection{Assegnamenti}
    \subsection{Forme Normali}
    \subsection{Naming}

\section{Logica del primo ordine}
    \subsection{Termini}
    \subsection{Formule}
    \subsection{Semantica}
    \subsection{Forme Normali}
    \subsection{Skolemizzazione}
    \subsection{Unificazione}

\section{Soddisfacibilità e Validità}

\section{Resolution}
\section{Il formato TPTP}

\end{document}