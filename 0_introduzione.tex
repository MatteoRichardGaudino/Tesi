\documentclass[./main.tex]{subfiles}

\begin{document}

\chapter*{Introduzione}
\addcontentsline{toc}{chapter}{Introduzione}

% \blindtext[2]

% Il problema della soddisfacibilità, i theorem prover, vampire, 
% i frammenti della logica del primo ordine, i binding fragments. 
% Strategie e metodi adottati. 

% Descrizione dei capitoli successivi.

Informatica e logica hanno sempre avuto molto in comune.
Entrambe nascono in qualche modo per descrivere e/o imitare il cervello e il ragionamento umano. 
Il sogno di Alan Turing infatti quello di "costruire un cervello" artificiale,
così come il sogno di molti logici è quello di "formalizzare il ragionamento umano".
Sebbene attualmente lo studio dell'intelligenza si basa principalmente su metodi di apprendimento 
piuttosto che su deduzione automatica, la logica rimane comunque un linguaggio estremamente utile
in grado di descrivere in modo preciso il linguaggio matematico.
La dimostrazione automatica di teoremi è un'importante branca dell'informatica e della logica
e trova largo utilizzo nell'ambito della verifica formale di software e hardware 
e i sistemi ATP (Automated theorem proving) sono utilizzati come strumento di supporto 
nella ricerca matematica per affinare o verificare dimostrazioni formali.
%
I primi tentativi di realizzare un sistema 
di dimostrazione automatica risalgono agli anni '50,
quando i ricercatori Allen Newell, Herbert A. Simon, and Cliff Shaw.
Progettarono il primo dimostratore automatico il \textit{Logic Theorist}
che, tramite varie euristiche, tentava di simulare il ragionamento umano. 
Da quel momento la ricerca si è spostata su metodi più formali e rigorosi,
meno human-friendly e più adatti ad essere eseguiti da un calcolatore.
Gli studi moderni si basano principalmente sugli studi 
del logico Herbrand e del metodo di Risoluzione di Robinson.
% 
Il problema della soddisfacibilità (SAT) è uno dei problemi più studiati in informatica.
Il problema consiste del determinare se una formula è soddisfacibile o meno ossia 
se esiste un'interpretazione che la rende vera.
Il problema è decidibile per la logica proposizionale, ma diventa indecidibile
per la logica del primo ordine.
Per questo la ricerca si è spostata sulla ricerca di frammenti sintattici della logica del primo ordine
decidibili rispetto al problema della soddisfacibilità.
Tra questi vi sono i frammenti Binding che sono oggetto di studio di questa tesi.
% 
Questa tesi si propone di implementare l'algoritmo di soddisfacibilità per i frammenti Binding
all'interno del sistema di dimostrazione automatica Vampire.
% 
Vampire è un moderno ATP creato da Andrei Voronkov e Alexandre Riazanov presso l'Università di Manchester.
Uno dei suoi punti di forza `e l’efficienza, Il team di sviluppo infatti partecipa
annualmente al CASC (The CADE ATP System Competition), una competizione tra sistemi ATP, e
fino ad ora ha sempre vinto almeno in una categoria ogni anno.
La sua implementazione è open-source e sviluppato in C++.
% 
La tesi è strutturata come segue:
\begin{itemize}
    \item Nel capitolo 1 verra data un'introduzione 
    alla logica proposizionale e del primo ordine, al problema della soddisfacibilità,
    ai teoremi di incompletezza di Gödel e al problema della dimostrazione automatica.
    \item Nel capitolo 2 verranno presentati i frammenti Binding e l'algoritmo di decisione.
    \item Nel capitolo 3 verranno descritte le componenti principali di Vampire con particolare attenzione
    a quelli necessari per l'implementazione dell'algoritmo di decisione per i frammenti Binding.
    \item Nel capitolo 4 verrà descritta l'implementazione dell'algoritmo di decisione per i frammenti Binding
    \item Nel capitolo 5 verranno presentati i risultati sperimentali ed un confronto con Vampire standard.
\end{itemize}



\end{document}