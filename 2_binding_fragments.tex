\documentclass[./main.tex]{subfiles}

\begin{document}
\chapter{Algoritmo di decisione di Frammenti Binding}
Nella sezione \ref{sec:sat_val}, 
sono stati esaminati i teoremi di Gödel e Church, 
mentre nella sezione \ref{sec:resolution} sono state viste alcune delle loro conseguenze. 
La logica del primo ordine è intrinsecamente indecidibile; tuttavia, 
è possibile identificare alcune sue componenti che risultano decidibili.
Queste componenti sono dette \textit{Frammenti} della logica del primo ordine.
Si pensi ad esempio ai risultati di Herbrand citati nella sezione \ref{sec:resolution}.
Se una formula non contiene funzioni ed è universalmente quantificata allora l'universo di Herbrand è finito e 
vi sono un numero finito di possibili instanziazioni ground. 
In questo caso determinare la soddisfacibilità di una formula di questo tipo
è riducibie al problema della soddisfacibilità proposizionale che è notoriamente decidibile.
In letteratura questo frammento è noto come \textit{Bernays–Schönfinkel Fragment}.
Altre esempi di frammenti decidibili sono il \textit{Monadic Fragment}, 
il \textit{Two-variable Fragment}, \textit{Unary negation fragment} e il \textit{Guarded Fragment}.
In questo capitolo verrà descritta una famiglia di frammenti relativamente recente chiamata \textit{Binding Fragments}.

\section{Tassonomia dei Frammenti Binding}
Si dice che una formula del primo ordine appartiene alla classe \textit{Boolean Binding} (BB) se generata dalla seguente grammatica:

$$ \varphi :=  \top \mid \bot \mid (\varphi \lor \varphi) \mid (\varphi \land \varphi) \mid \mathcal{P}(\psi) $$
$$ \psi := \rho \mid (\psi \lor \psi) \mid (\psi \land \psi) $$

Dove $\mathcal{P}$ è un prefisso di quantificatori e $\rho$ è una combinazione booleana di letterali che hanno come argomento 
tutti la stessa lista di termini. 
Una formula di questo tipo verrà chiamata con il nome $\tau$-Binding, dove $\tau$ indica la lista di termini comune.
Ad esempio sono $(f_1(x_1), f_2)$-Binding le formule: 
$p_1(f_1(x_1), f_2)$, $p_1(f_1(x_1), f_2) \lor \lnot p_3(f_1(x_1), f_2)$.
Per semplicità di scrittura è possibile omettere la lista di termini comune e posizionarla in notazione postfissa:

$$p_1(f_1(x_1), f_2) \lor \lnot p_3(f_1(x_1), f_2) \text{ diventa } (p_1 \lor \lnot p_3)(f_1(x_1), f_2)$$

Con $\mathcal{B}^\tau$ verrà indicato l'insieme di tutte le formule $\tau$-Binding.
Si definisce la funzione $term: \mathcal{B}^\tau \rightarrow T^n$ 
che associa ogni $\tau$-Binding alla sua lista di termini comune $\tau$.
Ad esempio $term((p_1 \lor \lnot p_3)(f_1(x_1), f_2)) = (f_1(x_1), f_2)$.
Verranno chiamati impropriamente $\tau$-Binding anche formule universalmente quantificate la cui matrice è un $\tau$-Binding.
In questo caso ci si riferisce esclusivamente alla matrice della formula eliminando i quantificatori.


I frammenti Binding possono essere ottenuti restringendo le regole di $\psi$:

\begin{itemize}
    \item Il frammento \textit{One Binding} (1B) viene ottenuto restringendo la seconda formula 
    a $\psi := \rho$
    \item Il frammento \textit{Conjunctive Binding} (CB o $\land$B) viene ottenuto restringendo la seconda formula 
a $\psi := \rho \mid (\psi \land \psi)$
    \item Il frammento \textit{Disjunctive Binding} (DB o $\lor$B) viene ottenuto restringendo la seconda formula 
    a $\psi := \rho \mid (\psi \lor \psi)$
\end{itemize}

Un'istanza particolare del frammento 1B è quando la formula non contiene quantificatori esistenziali.
Una formula 1B con soli prefissi universali viene detta del frammento \textit{Universal One Binding} ($\forall 1B$).

% TODO qualche esempio sui grafi e i database


% ----------------------------------- Soddisfacibilità -----------------------------------
\section{Soddisfacibilità dei frammenti Binding}
In questa sezione verrà analizzato il problema della soddisfacibilità dei frammenti binding.
In particolare verrà descritto l'algoritmo di decisione per i frammenti 1B e CB che è il soggetto principale dello studio di questa tesi.

Data una formula del frammento 1B è facile osservare che il processo di skolemizzazione
converte la formula in formato $\forall 1B$. 
Se si applica la stessa procedura ad una formula CB, le sottoformule generate dalla regola $\psi$ saranno del tipo:
$\mathcal{P} (\rho_1 \land ... \land \rho_n)$ con $\mathcal{P}$ un prefisso universale e $(\rho_1 \land ... \land \rho_n)$
$\tau$-Binding. In questo caso è possibile distribuire il '$\forall$' sui vari $\tau$-Binding e si ottiene così una formula 
equisoddisfacibile in formato $\forall$1B.



\newtheorem{thm}{Teorema: Decidibilità dei frammenti 1B e CB}[section]
\begin{thm}
    I frammenti 1B e CB sono frammenti decidibili del primo ordine.
\end{thm}

Una dimostrazione dettagliata di questo teorema può essere trovata nell'articolo [TODO da citare].
Si può osservare che il processo di clausificazione del primo ordine porta alla generazione di una formula equisoddisfacibile
che rispetta il formato DB. Ne consegue immediatamente per il teorema di Church:

\newtheorem{db}[thm]{Teorema: Indecidibilità del frammento Disjunctive Binding}
\begin{db}
    Il frammento DB è un frammento indecidibile del primo ordine.
\end{db}

\begin{proof}
Per assurdo Esiste un algoritmo di decisione totale $S$ per formule del frammento DB. Data una qualunque formula
$\varphi$ è possibile trasformarla in una equisoddisfacibile in formato CNF. Se si distribuisce il quantificatore 
universale sulle clausole si ottiene una formula $\varphi'$ che rispetta
i requisiti sintattici del frammento DB. $S$ è quindi una procedura di decisione totale per tutta la logica 
del primo ordine ma ciò è in contraddizione con il teorema di Church.
\end{proof}


Il processo di skolemizzazione consente di concentrarsi sullo studio del frammento $\forall$1B
per la risoluzione del problema della soddisfacibilità. 
Prima di descrivere l'algoritmo bisogna introdurre tre nuovi concetti:
L'Unificazione per $\tau$-Binding, Implicante di una formula del primo ordine e la conversione booleana di un $\tau$-Binding.
Data una formula del primo ordine $\varphi$ per Implicante di $\varphi$ si intende la conversione del primo ordine di
un implicante della 'struttura proposizionale esterna'.
ad esempio la formula
$\forall x_1 (p_1(x_1) \lor p_2(x_1)) \land (p_1(f_1) \lor \exists x_2 (p_3(x_2))) \land \lnot p_1(f_1) \land \exists x_2 (p_3(x_2))$
ha la seguente struttura booleana  $s_1 \land (s_2 \lor s_3) \land \lnot s_2 \land s_3$.
Un implicante (e anche il solo) di questa formula è l'insieme $\{s_1, s_3\}$ che ri-convertito nel primo ordine diventa
l'insieme $\{\forall x_1 (p_1(x_1) \lor p_2(x_1)), \exists x_2 (p_3(x_2))\}$.
In questo caso è stata creata implicitamente un bi-mappa tra costanti proposizionali e formule del primo ordine:
\begin{itemize}
    \item $s_1 \rightleftarrows \forall x_1 (p_1(x_1) \lor p_2(x_1))$
    \item $s_2 \rightleftarrows p_1(f_1)$
    \item $s_3 \rightleftarrows \exists x_2 (p_3(x_2))$
\end{itemize}
Un $\tau_1$-Biding e un $\tau_2$-Biding sono detti unificabili se e solo se l'insieme congiunto di tutti i loro letterali è unificabile.
Si può anche dire che sono unificabili sse 
le due liste $\tau_1$ e $\tau_2$ hanno la stessa lunghezza $n$ e dato un qualunque predicato $p$ $n$-ario
$p(\tau_1)$ e $p(\tau_2)$ sono unificabili. 
Una insieme di $\tau$-Biding è unificabile sse esiste una sostituzione che unifica a due a due tutti gli elementi dell'insieme.
Dato un $\tau$-Binding $\phi$ la sua conversione booleana $bool(\phi)$ è una formula proposizionale che si ottiene 
da $\phi$ mantenendo la sua struttura proposizionale, eliminando gli argomenti dai letterali e convertendo
i simboli di predicato in simboli di costante con lo stesso indice. Ad esempio il $\tau$-Binding
$((p_1 \land p_4) \lor p_2 \lor \lnot p_4)(\tau)$ viene convertito nella seguente formula proposizionale $(s_1 \land s_4) \lor s_2 \lor \lnot s_4$

A questo punto è possibile enunciare il teorema di caratterizzazione della soddisfacibilità del frammento $\forall$1B.

\newtheorem{forall1bsat}[thm]{Teorema: Caratterizzazione della soddisfacibilità per il frammento $\forall$1B}
\begin{forall1bsat}
    Data una formula $\varphi$ del frammento $\forall$1B, $\varphi$ è soddisfacibile se e solo se: \\
    Esiste un implicante $I$ dove: per ogni sottoinsieme $U \subseteq I$ di $\tau$-Binding,
    se $U=\{\gamma_1, ..., \gamma_n\}$ è unificabile allora la formula proposizionale $bool(\gamma_1) \land ... \land bool(\gamma_n)$ è soddisfacibile.
\end{forall1bsat}

Dal teorema appena descritto si estrapola intuitivamente l'algoritmo per la soddisfacibilità delle formule del frammento:

\begin{algorithm}[H]
    \caption{Algoritmo per la soddisfacibilità del frammento $\forall$1B}
    \KwSty{Firma:}{ oneBindingAlgorithm($\varphi$)}
  
    \KwIn{$\varphi$ una formula $\forall$1B}
    \KwOut{$\top$ o $\bot$}

    \ForEach{$I$ Implicant of $\varphi$}{
        $res := \top$\;
        \ForEach{$(U := \{\gamma_1, ..., \gamma_n\}) \subseteq I$}{
            \If{$U$ is unifiable}{
                \If{$bool(\gamma_1) \land ... \land bool(\gamma_n)$ is not satisfiable}{
                    $res := \bot$\;
                    \textbf{Break}\;
                }
            }
        }
        \If{$res = \top$}{
            \Return{$\top$}
        }
    }
    \Return{$\bot$}
  \end{algorithm}

I prossimi capitoli si concentreranno sullo studio dei dettagli tecnici per l'implementazione di questo algoritmo,
con annesse osservazioni sulle sfide implementative e una analisi dei risultati sperimentali ottenuti.




\end{document}