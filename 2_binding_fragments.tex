\documentclass[./main.tex]{subfiles}

\begin{document}
\chapter{Algoritmo di decisione di Frammenti Binding}
Come evidenziato nella sezione \ref{sec:sat_val}, 
sono stati esaminati i teoremi di Gödel e Church, mentre nella sezione \ref{sec:resolution} sono state approfondite le loro implicazioni. 
La logica del primo ordine è intrinsecamente indecidibile; tuttavia, 
è possibile identificare alcune sue componenti che risultano decidibili.
Queste componenti sono dette \textit{Frammenti} della logica del primo ordine.
Si pensi ad esempio ai teoremi di Herbrand citati nella sezione \ref{sec:resolution}.
Se una formula non contiene funzioni ed è universalmente quantificata allora l'universo di Herbrand è finito e 
vi sono un numero finito di possibili instanziazioni ground. 
In questo caso determinare la soddisfacibilità di una formula di questo tipo
è riducibie al problema della soddisfacibilità proposizionale che è notoriamente decidibile.
In letteratura questo frammento è noto come \textit{Bernays–Schönfinkel Fragment}.
Altre esempi di frammenti decidibili sono il \textit{Monadic Fragment}, 
il \textit{Two-variable Fragment}, \textit{Unary negation fragment} e il \textit{Guarded Fragment}.
In questo capitolo verrà descritta una famiglia di frammenti relativamente recente chiamata \textit{Binding Fragments}.



\section{Tassonomia dei Frammenti Binding}
\section{Algoritmo Astratto}

\end{document}